\chapter{Cloud computing}

Nejdříve ze všeho je potřeba vyjasnit pojem \uv{cloud computing}. Pro cloud computing zatím neujal žádný český ekvivalent. V článcích se většinou autoři spokojí na začátku s oznámením, že cloud znamená česky oblak nebo mrak a dále používají cloud computing popř. jenom cloud. 

\begin{quotation}
\uv{Přeci jenom \uv{výpočetní výkon v oblacích} by zněl divně (asi tak jako čistonosoplena), \uv{počítání mraků} zní jako počítání oveček, \uv{výpočetní/informační mrak} ještě hůře a \uv{život v oblacích} vyvolává zcela jiné představy (byť jde o významově nejbližší zpodobnění).}~\cite{termCloud}
\end{quotation}

V rámci této diplomové práce budeme používat termín cloud computing pro technologii a cloud pro označení produktů\todo{produktů zní divně}  založených na cloud computing.

\section{Definice}
\todo[noline, size=\small, color=green!40]{Název kapitoly: Definice je moc divná - historie asi nezahrnuje definici a "O" je moc krátký}
Cloud computing je definován jako služba, která je poskytována uživateli vzdáleně přes síť -- většinou Internet~\cite{defCloud}. To znamená, že není nutné mít všechno, co člověk používá, nainstalované a uložené u sebe na svém zařízení. Je možné mít data, aplikace nebo dokonce celý operační systém zpřístupněné na cloudu. 

Toto zavádí nové způsoby používání prostředků, které jsou přidělovány mnohem dynamičtěji tzn. je umožněno jednodušeji škálovat. Uživatel se nestará o hardware, který používá. Zprostředkovateli služby je zadáváno kolik a jakých prostředků budeme chtít využívat (např. velikost úložného prostoru nebo počet a výkon procesorů). Pokud je to možné jsou žádosti automaticky zpracovávány a vyřizovány. Název pro tento proces je on-demand self-servise, neboli samoobslužné vydávání prostředků na požádání.

Služby jsou dostupné přes jednotné API\footnote{Application Programming Interface -- rozhraní pro přístup k funkcím knihovny, programu nebo služby}, které je zpřístupněné přes internet. Koncový uživatelé využívají webové prohlížeče, mobilní aplikace a odlehčené programy pro stolní počítače. To umožňuje, aby celá infrastruktura byla všudypřítomná a uživatelé nebyli závislí na své poloze.

\todo[inline]{něco o placení?}


\todo[inline, color=red!50]{Motivace} 

\section{Historie}
Název vznikl podle častého znázornění Internetu na diagramech symbolem ve tvaru oblaku~\cite{defCloud2}. 

\todo[inline, color=red!50]{Historie}
\missingfigure[figwidth=\textwidth]{Obrázek vývoje - začátek - terminál a mainframe, osobní pc, zařízení a cloud \ldots}


\subsection{Druhy}
This is the first chapter of the thesis.

\section{Storage as a Service}

\subsection{Produkty}

\section{ownCloud}