\chapter{Úvod} 
Žijeme v době, ve které má čím dál více obyvatel přístup k internetu~\cite{InfTime}. Zvětšuje se podíl uživatelů, kteří přistupují na internet přes mobilní zařízení~\cite{InfTime}. Stává se obvyklým jevem, že jeden člověk má více mobilních zařízení. Například v roce 2011 bylo v České republice více než 120 mobilních telefonů na 100 obyvatel~\cite{InfTime}.

S rostoucím počtem zařízení, které mají přístup k internetu, se objevuje přání uživatelů, mít na všech svých zařízeních totožná data. Jednou z možností jak toto přání splnit je umístit data na jedno úložiště, které je dostupné ze všech zařízení.

Trendem v dnešní době je, podle zvyšujícího se objemu uložených objektů ~\cite{AmazonS3}, na cloud computing\footnote{\todo[inline, color=blue!40]{Krátká definice cloudu a odkaz na 1. kapitolu}} založené úložiště. 

V Plasma Active, uživatelském rozhraní pro zařízení s dotekovým ovládáním~\cite{PA}, je. 